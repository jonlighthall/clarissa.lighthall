
\noindent Printed by J. Wilber,\\
Lockport, N.Y.\\

\noindent %Held by \\
Yale Divinity Library\\
New Haven, Conn.\\

\noindent %Retrieved by\\
Interlibrary Loan Services\\
The Florida State University\\%

{\hspace*{-24pt} %to align with left margin
\begin{tabular}{r@{ }l}
Transcribed and annotated by Jonathan &Lighthall\\
son of James &Lighthall\\
son of William &Lighthall\\
son of Cone &Lighthall\\
son of Hiram &Lighthall\\
son of Lorenzo &Lighthall\\
son of Clarissa &Lighthall\\
\end{tabular}
}
\section*{Preface.}
\textsc{The} Brief Sketch of the Life of \textsc{Clarissa Lighthall}, which God taught her to commence at the age of fifteen\footnote{in the year 1816}, and through the great fear of losing her own soul she dare not omit it, although she accomplished it through much opposition and persecution.
It discovers that the rod of affliction caused her to humble herself at the foot of the cross, and to choose the better part which Mary chose, which never shall be taken from her, and to imitate the woman who washed Jesus' feet with her ears and wiped them with the hair of her head, in showing to the world around what a dear Savior she as found, and pointing to his redeeming blood, say behold the way to God; and also how the Prophets and Apostles felt when God discovered to them the Thunder of his Power and elected them to stand betwixt the living and the dead.---
As the visions of the Almighty were so terrible unto them, and the great conflict which they must shortly encounter was so set before them that their feeble frame could not endure it but they fell as dead men to the ground, so God has spoken to her a number of times by the revelation of His will unto her till there was no strength left within her---the terror of His Law from Sinai's burning mount, and such a woe was pronounced upon her, O it was beyond description to any but one that has experienced it!
If she did not preach the Gospel as He directed her; and she realized that affliction will either harden the sinner's heart like Pharaoh's, or soften it like that of the Child Jesus; she realized also that affliction cometh not forth from the dust, nor sorrow from the ground, and although affliction seems severe, yet in mercy it is sent to stop the Prodigal in his wild career and force him to repent and fly as from ruin's brink, and lay hold on the hope set before him, which is as an anchor to the soul, both sure and stedfast, and which entereth within the vail.

I write not this to please myself, but God.
I had rather hide this narrative in Oblivion's darkest shades than to disclose it to the world, but for Christ's sake, who was crucified and slain for me, and for the love of souls and my own, I endure this grevious cross.
I have prayed times without number that this bitter cup might pass from me, but God declares in His sacred writ that His will must be done or I can not be saved.

I expect that the evil spirits will condemn this book because they can not comprehend it nor understand it.
We behold the sun take its circuit from morn until eve, and we can not tell how it is, but we know it is so; thus it is with the dealing of God to my soul, and it is marvelous in our sight; but it is a Supreme Being that does it and I can not tell how.
When I used to peruse great experiences I did not doubt them nor condemn them because I had not arrived to such a high degree of grace and knowledge and glory of God.
Innumerable times have I realized the words of Prophet Jeremiah, whilst beholding sinners thronging the downward road to death and destruction---``O that my head were waters and my eyes a fountain of tears, that I might weep day and night for the slain of the daughter of my people;'' and Jesus' words oft times have I felt the efficacy of, when He wept over Jerusalem,---``O thou that stonest the Prophets, how often would I have gathered you as a hen gathereth her brood under her wings, but ye would not; and now your house is left to you desolate.''
This is left on record for our example---that we may not slight the offers of life and salvation; for there is a set time that Christ has opened a door of mercy to every soul, and O how happy it will be for those that enter in before it is forever closed against them!
O that every sinner would fear and tremble before His dreadful Majesty, and awake to righteousness and sin not, before God shall swear in his wrath that they shall not enter into His rest!
O, language is faint to describe the sympathy the Author of this book feels for perishing souls that are out of the Ark of Safety! and she feels to invite all the human race to come to the fountain which is opened in the house of King David for sin and unleanness, and wash and be clean.
O, she feels to entreat all to not only come and sip a little, but to drink a full supply, and bask in the Ocean of Love. and dwell forevermore in sweet Paradise regained by free pardon!
She feels to recommend this blessed religion as a substantial joy, a treasure untold, more precious than silver or gold, or all this earth can afford.
It constitutes these words true, that a contented mind is a continual feast, and it reconciles man to his lot.
A word to the Christian.---
Christ says, ``Be of good cheer, I have overcome the world;'' and ``It is your Father's good pleasure to give you the kingdom.''
The rougher the way the short their stay.
What need you care how goes the battle, if you gain the prize and win the field---if you only insure the end for which you were created, that is to know and glorify God here and enjoy him forever hereafter?
O then prove faithful until death, and you shall receive a crown of Life!
\begin{center}
A Christian is the highest state of man.\\
And is there who the blessed Cross wipes off\\
A foul blot from his dishonored brow?\\
If angels tremble, 'tis at such a sight.\\
A Deity believed is joy begun.\\
A Deity adored is joy advanced,\\
A Deity beloved is joy matured.\\
Behold what great things God has wrought for a poor worm of the dust!\\
Christ has sent me to invite you\\
To a rich and costly feast;\\
Let not doubts nor fears distress you,\\
Come, the rich provision taste!
\end{center}

\section*{Life of Clarissa Lighthall.}
I was born in the Town of Galway, County of Saratoga, State of New York, in the year 1801, October 24.
At the age of six years\footnote{in the year 1807} serious thoughts began to pass through my mind.
One of my neighbors died.
I went to see him.
O what a solemn sight!
One whom I loved, now a lifeless lump of clay.
The grim messenger of death was present before me for a long time.
I went up chamber to view his grave; a cloud passed over my mind; I viewed the shortness of time; a few more years and I must die!
Such was my ignorance, I did not know that I had a soul to be saved or lost.
I lamented that I could not see people any more; I thought I should sink into nonentity.
There were no Sunday Schools where I resided.
My parents' native place was in Connecticut.
My father\footnote{Gideon Draper, 1769--1850} was not a professor of religion; because he could not comprehend the birth of the Savior, or the plan of salvation, he would not imbask in the ocean of love.
He often said, if he could overcome his passion, he could live as good a christian as any one; he did not realize that naught but the grace of God can overcome the carnal mind.
Christ is the door, and he that seeketh to climb up some other way is a thief and a robber.
Our righteousness is but filthy rags, and we must cast them away and put on Christ's white robe of righteousness by faith and prayer.

My father was a Pettifogger; he often went to courts, and conversed much about the lawyers, and sometimes on the Scriptures; he mentioned the names of St.\ Paul and St.\ Peter; so ignorant was I of the Scriptures that I thought they were lawyers.
In my mother's\footnote{Dierus Hollister, 1765--1843} youthful days she lived a moral life, and attended the Quaker and Presbyterian meetings, but at the age of forty-two\footnote{in the year 1807} she was convinced that dead morality would never save her; she espoused the glorious cause of Christ, and joined the Methodist Episcopal Church.
I believe God ordered it thus for the salvation of her children; she said if she had resided near those other churches, she would have joined them.---
She taught our little feet early to resort to the House of God, and tread the holy courts with a song of praise; praise and extol the King of Heaven for it! bless his holy name! he takes the poor, foolish things of this earth to confound the wise and noble.
There was an unlearned man who went from house to house to warn his neighbors to flee from the wrath to come, and appointed prayer meetings, and exhorted, and prayed, and used all the energy that he could for the salvation of souls.
My soul was powerfully convicted many times, so that I wept and wished him to tell me how to get religion, but no one supposed I had any serious thoughts, for I could not tell my feelings until Jesus unlocked my heart and opened the visions of Heaven to my soul.

I did not go to school after the age of twelve years until I was seventeen\footnote{between 1813--1818}; I went four weeks, so I got but little learning; I might have went more but I did not want to go, and my parents did not make me.
I have often regretted my folly, when I see the necessity of more learning.
At the age of seven years\footnote{in the year 1808}, my father bought another farm and tavern stand on a four corners in the town of Broadalbin, Montgomery county\footnote{now Fulton County}, about fourteen miles\footnote{18.0 miles} from Ballston Springs\footnote{incorporation as the village of of Ballston Spa in 1807}, and the same distance from Johnstown\footnote{9.4 miles}, and removed there, about a half a mile form the other place.
Before this time he was worth about five thousand dollars.
He pulled down buildings and built larger, and began to sell the poison cup of liquor, to outdo the merchant on the other corner; and he lost some property by fire, so instead of increasing his property, he decreased it, to his sad disappointment.
He got his property by the trade of saddle and harness making.
I believe it is a great sin in the sight of God to retail out liquor to people; it injured my father much, and he gave it up, and the merchant broke.
I believe the curse of God will follow such crimes, although some men think it no harm.
The sight of my eyes affected my heart with great grief; I saw a young man get so intoxicated that he fell into the fire, and if my mother had no been in the room he would have burnt to death.
He might as well have died then, for it brought him to an untimely death; in the time of the little war, he enlisted three times and took his bounty, and was shot with five others.
A minister of the Gospel offered him a Bible to read and exhorted him to make his peace with his God; he disdained it, and said he preferred a bottle of rum.---
His mother was a pious Baptist woman; she was warned of his death by hearing a report of a gun over his bedroom.
She warned him not to go to the army, and to refrain from the error of his ways, and turn in with the offers of life and salvation, but instead of thanking her for her maternal advice, in a rage he cursed and swore at her.
His father was a notorious wretch; he lived so near a Methodist meeting house that he could sit in his door and hear them preach and pray, and shout the high praises of God.---
He was so angry at them, he said because they made such a noise, he cursed them and wished the house burned up; he perfectly hated God and His people.
Jesus saith it is better that a millstone should be hanged about the neck of any one and be sunk in the bottom of the sea, than to offend one of these little ones.
He declares that His children are as tender to him as the apple of His eye, and also the very hairs of their heads are all numbered, and not one falls unnoticed by His all-seeing eye.
It is strange that sinners do no consider that when they are injuring God's people they are doing the same to God, and He will punish them ere long.
Another son died in the same war; he was so insane that he cursed the war, and wandered about the fields, talking to the stumps and not knowing what he was about.
Thus we see the Judgments of God falling on this wretched man because he would not obey God and keep His commandments, in loving and serving Him, and for not bringing up his children in the nurture and admonition of the Lord, and when they are old they will not depart from it.
There he had the glorious sound of the Gospel so near him, but he rejected it, and would none of His counsel and therefore God mocked at his calamity and laughed when his fear came.
O how many righteous would have rejoiced at that high privilege while residing in distant lands far from the House of God!
The Word of the Lord saith, ``The wicked shall not live out half their days;'' and also, ``He that being often reproved, hardeneth his neck, shall suddenly be destroyed, and that without remedy.''
These passages of Scripture were fulfilled in poor Mr. Y------'s family; a number of his children were suddenly cut off by death, without a hope in Christ, and thus destroyed where hope or mercy never can reach them.

Two more painful circumstances I shall briefly mention, which tore my heart with grief so that I wished there was no liquor in the world.
I saw two nice females wasted away with grief because their companions gave themselves to intemperance to such a degree, and robbed them and their children of the comforts of life.
O how my heart rejoiced when I saw the Temperance Banner hoisted and the black flag of drunkenness going down! for I had beheld innumerable inhabitants slain thereby.
I had a long time desired that something might be done to prevent this great evil in the world.
In 1813, where I resided, there was a scarcely one man, young or old, but would get intoxicated; but now, Praise the Lord! how changed the scene! it is a shame for young or old to be seen tasting a drop of liquor, unless for medicine.

The first it came into my mind there was a Heaven o'er yonder sky where pleasure never dies, there was an infant departed this life; its sister was one of my playmates, and while she was mourning its loss, I said to her, ``Do not mourn for Daniel; for he has gone to Heaven, and when you die you will see him again.''
I loved every body, and thought all mankind was good; I did not know the fallen nature of man by sin.---
Children's minds are opened very young to receive instruction of a spiritual nature.
At the age of ten and twelve\footnote{in the years 1811 and 1813} many things occurred which warned me to prepare for death; I dreamed that I saw Hell with all its frightful appearance; I saw myself a great sinner; my sins were all set in array before me; I dare scarcely go to sleep for fear I should wake up in those tormenting pains where the worm dieth not and the fire is not quenched.
Two of my schoolmates were suddenly cut off by the cold frosts of death.
I attended one of the funerals, and wept bitterly on account of the solemnity of death.
My mother asked some one what ailed me; I could not express my feelings.
Mr.\ James came to the schoolhouse and exhorted us that we must die also, and all the scholars' minds were affected so that they wept much.
I was very fond of play and amusement like all other children, and those serious thoughts would soon wear off.
My father was a very profane man when he was in a passion; so when I got angry at my work, when it did not go well, I would follow his bad example, but I dare not let him know it, for he saw the evil of it and did not allow his children in it.

I was always very worldly minded.
I loved to work as well as eat, and fulfilled that passage of Scripture, ``Whatsoever thy hand findeth to do, do it with thy might.''
After doing the task that my mother set me, I went to a chestnut ridge and picked up chestnuts enough to buy me a brass kettle when I was married.
Instead of spending my money for foolish things, I lent it my father.
It cost five dollars and two shillings.
I spent some money for two pair of ear-rings, and afterwards wept because I ought to have put it to some better use.
I spun in three years sixty yards of tow and linen cloth, and at fourteen years of age\footnote{in 1815} I wove it, besides helping my mother about housework.

My father build a large tavern, one whole side of which was a large ball room, and they had balls and shows and singing schools, which diverted my mind for a short time; I also went to parties a little, but something would occur that would leave a sting behind, and I would resolve never to go again; for daily wisdom taught me that there is no substantial joy to be found in all the gay flowers of Nature's garden---soon it would all fade like a blasted rose, for death was on my track.
I erected an idol and worshiped it, and it was destroyed on God's holy Sabbath day, and I was glad that it was gone.---
This sin was a great grief to me for may years.
One time I worked a little on the Sabbath, and I felt so guilty that I imagined I saw the Devil coming after me, and I ran out of the room; in every place where I worshiped that idol God me em worship Him afterwards, and I give glory to His name for it.
The Lord showed me in a wonderful manner that the same Fiend of Darkness designed to destroy me as he did Moses, Joseph and the Child Jesus, but God prevented it because he knew that He was going to raise me up to some great work.
Father said he did not desire any female children, for they were no profit.
God saw fit to give him a daughter\footnote{Harriet Draper, 1796--1834}, and then a son\footnote{Harrison Draper, 1797--1798}; and at the age of twenty months He saw best in His wise Providence to bereave him of his only son; I presume, to punish him for setting his heart on males more than females.
The next was a daughter\footnote{Maria Draper, 1800--1883}, which grieved him much but she proved to be a very beautiful child, and he always called her his handsome girl after she was grown up.
The next child he was very sure would be a son, but to his great disappointment and sorrow it was a daughter\footnote{The Author, Clarissa Draper, 1801--1874}.
The other two children\footnote{Harriet and Maria} had the whooping cough, and he told them to go into the room that I might have it and die, but my mother would not allow it.
I was dark complexioned, like my mother, and all the other children light, like my father; so I, like Leah, was hated.
My father would call me his black girl, which made me feel very bad, and would make me work in the field.
I undervalued myself, and was often so discouraged that I wished myself dead.
He would praise me, and promise me a new dress, and said that I could do as much work as any man he could hire, and mourn that I was not a boy.
He said that I had more life and ambition than both of his sons, but he never gave me a dress in his life.
He raise two sons\footnote{Hiram Draper, 1804--1841; and Aheimas Draper, 1808--1888} and four daughters\footnote{Harriet, Maria, and Clarissa; plus Laura Draper, 1806--1900}; but strangers respected me ore than the rest of the children; when travelers called at the tavern, they would choose me to do errands and give me money.
I grew so fast that I was just as large as my sister elder\footnote{This is presumably a reference to Maria, given that Maria was only a year older than Clarissa.}.
Some said I was the handsomest; other said she was; there was a merchant's lady replied that she wished her girls would work for her as I did for my mother, but she never expected it.

I bought a book of a peddler, read it through, and swapped it a number of times, until I came to one I loved much; the title of ti was, ``\textit{The Life and Deaths of Eighteen Children, who experienced Religion and died happy in the Lord.}''
I would go into sweet solitude and read and weep over myself and human woes.
I could never bear to see one child hurt or hector another.
I never delighted in any know sin; as soon as I found that I had done wrong it grieved me.
If any child refused to give me any thing that I wanted, if I had any ting that they desired, I would give it them.
I could never lay up revenge against any one, but would forget and forgive, let the injury be ever so great.
``Vengeance is mind, saith the Lord; I will repay it.''

When my father gave up keeping a Public house, he turned it into a meeting house, and a place for Preachers to make their home; when there were Quarterly Meetings at the meeting house about a mile off, he would keep about twenty or thirty over night.---
O the happiness I enjoyed then with the people of God!
I loved them as my own soul.
My father let mother take the team, but would no go himself; she would get some Methodist man to drive, and take her children and as many poor women that had no other way to go, as she could.
Sometimes we would go ten miles or more.
The first one I recollect attending they had a blessed time---a shout in the camp of Israel!
I took my bows off my bonnet, and intended to go in to the love-feast; the people were not allowed to go in those times ad they do at present.
My mother took me by the hand and led me to the door, and the Minister began to talk to me; I was so foolish that I did not want to hear him, and went away and did not go in.
There was a very pious girl sung very sweet and melodious; it will never be erased from my memory; how her face shined like an angel of light!
I attended the preaching meeting; the house was crowded and many could not get in; some wept and clapped their hands and shouted, ``Glory to God in the  highest! others fell with the power of the Lord to the floor, and some were convicted and converted.
I trembled like Belshazzar, and there was a young lady dressed in deep mourning, who was siezed with the power of the Most High, so that she seemed lifeless; her wicked brothers were very angry,and resolved to take her by force out of the house; the Preachers talked to them, but they would not use reason, but acted like fools.
After they got her out of the door a young female held her up, and she, with the tears trickling down her cheeks, exhorted sinners to flee from the wrath to come, while there was a great multitude thronging around her.
The solemnity of this event was beyond the power of human tongue to describe; so deeply was it impressed on my mind that it never could be effaced; for at that present moment it is just as bright to my view as when I saw it.
Such preaching as that---it outpreached all the Ministers in the Pulpit, because there was a God in it; it was not flesh and blood that performed this work.
Glory to God! my soul is filled with the heavenly manna while I write.---
It makes the Devil dreadful mad when God performs His powerful work.
He will work and none can hinder, praise Hist holy name!
He has all power both in heaven and earth, and whom He will He cast down, and whom He will He raises up.

My eldest sister\footnote{Harriet} gave herself up to merry mirth and vanity; if she had any serious thought, she cast them all away.
She would go to all the ball and parties, and dance and play, in spite of all our folks could say.
Mr.\ Jones conversed with her about the salvation of her soul; he told her she might die in her youthful days like many others, and she ought to prepare to meet her Creator.
Tobias Spicer traveled that Circuit that year; he came to our house, and conversed with both my eldest sisters about their eternal welfare, and brought those solemn verses and sung them in the most solemn manner:---
\begin{quote}
\setlength{\parindent}{2ex}
    Young people! all attention give,
    \setlength{\parskip}{0pt}
    \par While I address you in God's name;\\
    You, who in sin and folly live,
    \par Come, hear the counsel of a friend.
\end{quote}

My sister next to the eldest\footnote{Maria} chose God for her portion, and a number of her playmates also united themselves to the Methodist Episcopal Church, and walked a holy life, but the other\footnote{Harriet} chose the world and its vanities for her portion, and at the age of eighteen\footnote{in the year 1814} was married to a Baptist preacher's son\footnote{Joseph Perkins, 1793--1870, aged 21 at the time, was son of Nathaniel Perkins}, and grief and sorrow were multiplied unto them in a great measure.
He was a professor of religion, but did not adorn it by a well ordered life and godly conversation; he was very light and trifling, and used many jocose words, which are forbidden in the Scriptures.
He was not worth a cent, but was steady and industrious.
He was a sweet singer, and a good musician; he sang those solemn words, (which made a great and lasting impression on my mind, for I awfully feared that if I did not repent, I should share her fate)---
\begin{quote}
    \begin{center}
        POOR POLLY'S FATE.
    \end{center}
    Young people: listen, while I relate\\
    The store of poor Polly's fate;\\
    She would go to ball, and dance and play,\\
    In spite of all her friends could say;\\
    She said she would turn when she grew old,\\
    And God would then receive her soul.\\
    One Friday morning she grew sick---\\
    Her stubborn heart began to break;\\
    She called her mother to her bedside---\\
    She wrung her hands---she groaned, she cried,\\
    She gnawed her tongue before she died.\\
    She said--``Weep not fore me! but remember well,\\
    When wicked Polly's dead, she groans in hell!''\\
    It almost broke her parents' heart,\\
    To think their daughter must depart;\\
    To think with devils she must be,\\
    In hell, to all eternity.\\
\end{quote}

My father was very much disgusted because my sister\footnote{Maria} joined the Methodist church---his beautiful girl, that he doted so much on.
He calculated to make her a gay flower in the circle of life; he got a fiddler to come to his house to learn her to dance, but now all his blooming hopes were cut off.
The Methodists were counted as the Apostles said they were, the filth and offscouring of all things\footnote{This wording is a reference to Corinthians 4:13. The same wording is used to describe Methodists in the Memoir of Mrs.\ Maria Gordon appearing in the Wesleyan-Methodist Magazine, March 1835}.
Both our temporal and spiritual births were just alike---there was about two years odds in our ages\footnote{this suggests that Maria was born in early 1800.}; we both experienced religion in the fifteenth year of our age\footnote{in 1815 and 1816, respectively}.
She reproved me because I was not more serious; she said that when she was at my age she was more concerned bout her soul's salvation than I was.
She did not know the strivings of the Holy Spirit that i endured.
I, Mary-like, pondered them all in my hart, and told no one.

My mother taught her children to learn hymns, and when she heard any of them speak evil words about any one, she would reprove us, and repeat Pope's ``Universal Prayer'' to us:---
\begin{quote}
\setlength{\parindent}{2ex}
    Teach me to feel another's wo,
    \setlength{\parskip}{0pt}
    \par And hide the faults I see;\\
    The mercy I to others show,
    \par That mercy show to me.
\end{quote}

I learned the Doxology the first:---
\begin{quote}
    Praise God! from whom all blessings flow;\\
    Praise Him! all creatures here below;\\
    Praise Him! above, ye heavenly host;\\
    Praise Father, Son and Holy Ghost!
\end{quote}

My father had three cousins that were Methodist Preachers, and they were stars of the first magnitude, that shined bright with the glory of God, and their gifts and talents were great, and they were endowed with wisdom from on high.
On was a Presiding Elder; he made very much of me; he had a very winning way; he gained the affections of almost every one that knew him.
