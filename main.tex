\documentclass{article}
\usepackage[utf8]{inputenc}

\title{The Life of Mrs.\ Clarissa Lighthall.}
\author{Written by herself.}
\date{1854.}

\begin{document}
\maketitle

\noindent Printed by J. Wilber,\\
Lockport, N.Y.\\

\noindent %Held by \\
Yale Divinity Library\\
New Haven, Conn.\\

\noindent %Retrieved by\\
Interlibrary Loan Services\\
The Florida State University\\%

{\hspace*{-24pt} %to align with left margin
\begin{tabular}{r@{ }l}
Transcribed by Jonathan &Lighthall\\
son of James &Lighthall\\
son of William &Lighthall\\
son of Cone &Lighthall\\
son of Hiram &Lighthall\\
son of Lorenzo &Lighthall\\
son of Clarissa &Lighthall\\
\end{tabular}
}
\section*{Preface.}
\textsc{The} Brief Sketch of the Life of \textsc{Clarissa Lighthall}, which God taught her to commence at the age of fifteen, and through the great fear of loser her own soul she dare not omit it, although she accomplished it through much opposition and persecution.
It discovers that the rod of affliction caused her to humble herself at the foot of the cross, and to choose the better part which Mary chose, which never shall be taken from her, and to imitate the woman who washed Jesus' feet with her ears and wiped them with the hair of her head, in showing to the world around what a dear Savior she as found, and pointing to his redeeming blood, say behold the way to God; and also how the Prophets and Apostles felt when God discovered to them the Thunder of his Power and elected them to stand betwixt the living and the dead.---
As the visions of the Almighty were so terrible unto them, and the great conflict which they must shortly encounter was so set before them that their feeble frame could not endure it but they fell as dead men to the ground, so God has spoken to her a number of times by the revelation of His will unto her till there was no strength left within her---the terror of His Law from Sinai's burning mount, and such a woe was pronounced upon her, O it was beyond description to any but one that has experienced it!
If she did not preach the Gospel as He directed her; and she realized that affliction will either harden the sinner's heart like Pharaoh's, or soften it like that of the Child Jesus; she realized also that affliction cometh not forth from the dust, nor sorrow from the ground, and although affliction seems severe, yet in mercy it is sent to stop the Prodigal in his wild career and force him to repent and fly as from ruin's brink, and lay hold on the hope set before him, which is as an anchor to the soul, both sure and stedfast, and which entereth within the vail.

I write not this to please myself, but God.
I had rather hide this narrative in Oblivion's darkest shades than to disclose it to the world, but for Christ's sake, who was crucified and slain for me, and for the love of souls and my own, I endure this grevious cross.
I have prayed times without number that this bitter cup might pass from me, but God declares in His sacred writ that His will must be done or I can not be saved.

I expect that the evil spirits will condemn this book because they can not comprehend it nor understand it.
We behold the sun take its circuit from morn until eve, and we can not tell how it is, but we know it is so; thus it is with the dealing of God to my soul, and it is marvelous in our sight; but it is a Supreme Being that does it and I can not tell how.
When I used to peruse great experiences I did not doubt them nor condemn them because I had not arrived to such a high degree of grace and knowledge and glory of God.
Innumerable times have I realized the words of Prophet Jeremiah, whilst beholding sinners thronging the downward road to death and destruction---``O that my head were waters and my eyes a fountain of tears, that I might weep day and night for the slain of the daughter of my people;'' and Jesus' words oft times have I felt the efficacy of, when He wept over Jerusalem,---``O thou that stonest the Prophets, how often would I have gathered you as a hen gathereth her brood under her wings, but ye would not; and now your house is left to you desolate.''
This is left on record for our example---that we may not slight the offers of life and salvation; for there is a set time that Christ has opened a door of mercy to every soul, and O how happy it will be for those that enter in before it is forever closed against them!
O that every sinner would fear and tremble before His dreadful Majesty, and awake to righteousness and sin not, before God shall swear in his wrath that they shall not enter into His rest!
O, language is faint to describe the sympathy the Author of this book feels for perishing souls that are out of the 

\section*{Life of Clarissa Lighthall.}
I was born in the Town of Galway, County of Saratoga, State of New York, in the year 1801, October 24.
At the age of six years serious thoughts began to pass through my mind.
One of my neighbors died.
I went to see him.
O what a solemn sight!
One whom I loved, now a lifeless lump of clay.
The grim messenger of death was present before me for a long time.
I went up chamber to view his grave; a cloud passed over my mind; I viewed the shortness of time; a few more years and I must die!
Such was my ignorance, I did not know that I had a soul to be saved or lost.
I lamented that I could not see people any more; I thought I should sink into nonentity.
There were no Sunday Schools where I resided.
My parents' native place was in Connecticut.
My father was not a professor of religion; because he could not comprehend the birth of the Savior, or the plan of salvation, he would not imbask in the ocean of love.
He often said, if he could overcome his passion, he could live as good a christian as any one; he did not realize that naught but the grace of God can overcome the carnal mind.
Christ is the door, and he that seeketh to climb up some other way is a thief and a robber.
Our righteousness is but filthy rags, and we must cast them away and put on Christ's white robe of righteousness by faith and prayer.
\end{document}
