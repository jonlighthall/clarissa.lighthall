\documentclass{article}
\usepackage[utf8]{inputenc}

\title{The Life of Mrs.\ Clarissa Lighthall.}
\author{Written by herself.}
\date{1854.}

\begin{document}
\maketitle

\noindent Printed by J. Wilber,\\
Lockport, N.Y.\\

\noindent %Held by \\
Yale Divinity Library\\
New Haven, Conn.\\

\noindent %Retrieved by\\
Interlibrary Loan Services\\
The Florida State University\\%

{\hspace*{-24pt} %to align with left margin
\begin{tabular}{r@{ }l}
Transcribed by Jonathan &Lighthall\\
son of James &Lighthall\\
son of William &Lighthall\\
son of Cone &Lighthall\\
son of Hiram &Lighthall\\
son of Lorenzo &Lighthall\\
son of Clarissa &Lighthall\\
\end{tabular}
}
\section*{Preface.}
\textsc{The} Brief Sketch of the Life of \textsc{Clarissa Lighthall}, which God taught her to commence at the age of fifteen, and through the great fear of loser her own soul she dare not omit it, although she accomplished it through much opposition and persecution.
It discovers that the rod of affliction caused her to humble herself at the foot of the cross, and to choose the better part which Mary chose, which never shall be taken from her, and to imitate the woman who washed Jesus' feet with her ears and wiped them with the hair of her head, in showing to the world around what a dear Savior she as found, and pointing to his redeeming blood, say behold the way to God; and also how the Prophets and Apostles felt when God discovered to them the Thunder of his Power and elected them to stand betwixt the living and the dead.---
As the visions of the Almighty were so terrible unto them, and the great conflict which they must shortly encounter was so set before them that their feeble frame could not endure it but they fell as dead men to the ground, so God has spoken to her a number of times by the revelation of His will unto her till there was no strength left within her---the terror of His Law from Sinai's burning mount, and such a woe was pronounced upon her, O it was beyond description to any but one that has experienced it!
If she did not preach the Gospel as He directed her; and she realized that affliction will either harden the sinner's heart like Pharaoh's, or soften it like that of the Child Jesus; she realized also that affliction cometh not forth from the dust, nor sorrow from the ground, and although affliction seems severe, yet in mercy it is sent to stop the Prodigal in his wild career and force him to repent and fly as from ruin's brink, and lay hold on the hope set before him, which is as an anchor to the soul, both sure and stedfast, and which entereth within the vail.

I write not this to please myself, but God.
I had rather hide this narrative in Oblivion's darkest shades than to disclose it to the world, but for Christ's sake, who was crucified and slain for me, and for the love of souls and my own, I endure this grevious cross.
\end{document}
